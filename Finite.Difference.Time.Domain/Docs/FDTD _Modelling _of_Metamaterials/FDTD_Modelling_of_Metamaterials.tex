\documentclass{report}
\usepackage{pgf}
\usepackage{tikz}
\usepackage{verbatim}
\usepackage{url}
\usepackage{hyperref}	% Clickable links to figures, references and urls.

\usepackage{graphicx}

\begin{document}

\title{Finite Difference Time-Domain Modelling of Metamaterials: A GPU-Based Implementation of Lossless Cylindrical Cloak}
\author{Attique Dawood}
\maketitle

\tableofcontents
\listoffigures

\begin{abstract}
The Finite Difference Time-Domain method is a differential numerical technique for solving electromagnetic wave scattering problems. In order to model metamaterials with negative refractive index the standard algorithm is modified as implementation of the Drude dispersion model. Suitable values of $\omega_p$  are chosen to model the metamaterial cloaking structure following the treatment of~\cite{Radial-Zhao}. The lossless case of electromagnetic cloak is implemented using a graphics processing unit (GPU).
\end{abstract}

\chapter{Introduction}
\section{The Finite Difference Time-Domain Method}

FDTD is a differential technique for modelling electromagnetic problems

\section{Stability and Causality}

The FDTD method is constrained by the stability relation known as the Courant stability criterion.

\section{The Yee Algorithm}

The original algorithm was proposed by K. S. Yee in 1966~\cite{Yee1966}.

\chapter{Dispersion Models for FDTD}

\section{Dispersive Models}
Debye, Drude and Lorentz dispersion models.

\section{Using the Drude Dispersive Model}
Using Drude model values of plasma frequencies for regions corresponding to given values of $\epsilon$ and $\mu$ are specified. The FDTD methods requires some modification~\cite{Radial-Zhao}.

\chapter{Problem Specification}

\begin{figure}[here]
\centering
\includegraphics[width=\textwidth]{Algorithm.png}
\caption{The Yee algorithm. Figure taken from~\cite{JBSchneiderUFDTD}}
\label{Algorithm}
\end{figure}

\chapter{Solution Approach}



\section{Implementation Details}


\nocite{*}
\bibliographystyle{ieeetr} %plain, ieeetr
\bibliography{FDTDref}

\end{document}




