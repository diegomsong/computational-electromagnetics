\documentclass{report}
\usepackage{amsmath}
\usepackage{pgf}
\usepackage{tikz}
\usepackage{verbatim}
\usepackage{url}
\usepackage{hyperref}	% Clickable links to figures, references and urls.

\usepackage{graphicx}

%Chapter 1 One Dimension
%1. 1D free space
%2. 1D homogeneous scatterer epsilon=2
%Chapter 2 Two Dimension
%3. 2D free space
%4. 2D cylinder problem from sadiku
%Chapter 3 Dispersive Media
%5. 1D scatterer using Drude, epsilon=2
%6. 2D dispersive media, cylinder problem from Sadiku using Drude
%Chapter 4 Metamaterial Modelling using Dispersive Drude Model
%7. 1D DNG slab
%8. 2D DNG slab
%Chapter 5 FDTD Modelling of Lossless Cylindrical Cloak
%9. 2D FDTD implementation on Matlab
%10. 2D FDTD implementation on C++
%Chapter 6 GPU implementation of Lossless Cylindrical Cloak
%11. GPU considerations
%12. Implementation details
%13. Performance Analysis
%Conclusion

\begin{document}

\title{Finite Difference Time-Domain Modelling of Metamaterials: A GPU-Based Implementation of Lossless Cylindrical Cloak}
\author{Attique Dawood}
\maketitle

\tableofcontents
\listoffigures

\begin{abstract}
The Finite Difference Time-Domain method is a differential numerical technique for solving electromagnetic wave scattering problems. In order to model metamaterials with negative refractive index the standard algorithm is modified as implementation of the Drude dispersion model. Suitable values of $\omega_p$  are chosen to model the metamaterial cloaking structure following the treatment of~\cite{Radial-Zhao}. The lossless case of electromagnetic cloak is implemented using a graphics processing unit (GPU).
\end{abstract}

\chapter{Introduction}
\section{Metamaterials}
Metamaterials are artificially manufactured with certain structural geometry. Electromagnetic waves passing through these metamaterials will act differently compared to naturally occurring materials. The behaviour of these metamaterials can change with frequency of incident EM wave and may exhibit negative values of permittivity and permeability under certain conditions. Some applications of metamaterials are perfect lens, invisibility cloak and novel antenna designs.

\section{Modelling Techniques}
Metamaterials can be modelled using analytical or numerical techniques. For problems with simple geometry and symmetry analytical methods give exact solution. For more complex structures of geometry numerical techniques can be applied. Numerical techniques in electromagnetics are generally classified as either differential or integral. Examples of integral techniques are method of moment (MoM) and finite element method (FEM), whereas, finite difference time-domain (FDTD) is a differential technique.

\section{FDTD Modelling}
Integral techniques usually involve solving for unknowns in the problem domain. The number of unknowns determine the computational time required. On the other hand, FDTD is a brute force method that requires the whole problem domain to be discretized and stored in computer memory. Thus, the amount of computer memory is a constraint in the case of FDTD. Conceptually, FDTD is easier compared to integral methods.

\section{FDTD on Modern Hardware}
In recent times computer memory has become extremely cheap and a large number of complex problems can be readily and easily modelled using FDTD. Moreover, the recent advent of multi-core CPUs (central processing unit) favour methods and algorithms that can take advantage of parallel processing. Since, FDTD is a data parallel algorithm it is ideal in such a scenario.

In more recent times, GPUs (graphics processing units) with general purpose computing capabilities have arrived on the scene. These GPGPUs are in the order of 40--50 times faster than contemporary CPUs and are heavily multi-threaded. The goal of this thesis is to employ GPGPU and model the classical cylindrical cloak using the FDTD method proposed in~\cite{Radial-Zhao}.

%\chapter{FDTD Modelling of Metamaterials}
\chapter{FDTD Modelling in One Dimension (1D)}

\section{Introduction to FDTD: The Yee Algorithm}

The original algorithm was proposed by K. S. Yee in 1966~\cite{Yee1966}. The derivatives in Faraday's Law and Ampere's Law are replaced with difference equations. The whole problem space is discretized and divided into cells such that electric and magnetic fields are staggered at half spatial steps in unit cell. By advancing the simulation in time, future values of electric and magnetic field components are computed using past values. This chapter gives provides a brief treatment of FDTD in 1D and 2D. Wave propagation in free space and with a homogeneous scatterer are discussed using conventional and dispersive FDTD methods.

\section{FDTD Update Equations in 1D}
Faraday's Law and Ampere's Law in differential form are given by:
\begin{equation}
\centering
\nabla \times \textbf{E} = - \dfrac{\partial \textbf{B}}{\partial t}
\label{Faraday's Law}
\end{equation}

\begin{equation}
\centering
\nabla \times \textbf{H} = \dfrac{\partial \textbf{D}}{\partial t}
\label{Ampere's Law}
\end{equation}

Assume electric field only has $x$-component, that is $E_x$, and magnetic field only has $y$-component $H_y$. The curl of electric field can be expanded as:
\begin{equation}
\centering
\nabla \times \textbf{E} = - \mu \dfrac{\partial H_y}{\partial t} \hat{y} = \left| \begin{array}{ccc} \hat{x} & \hat{y} & \hat{z} \\ \frac{\partial}{\partial x} & \frac{\partial}{\partial y} & \frac{\partial}{\partial z} \\ E_x & 0 & 0 \end{array} \right| = \dfrac{\partial E_x}{\partial z} \hat{y}
\label{Faraday's Law-Curl-Expansion}
\end{equation}
Similarly, the curl of magnetic field can be written as:
\begin{equation}
\centering
\nabla \times \textbf{H} = \epsilon \dfrac{\partial E_x}{\partial t} \hat{x} = \left| \begin{array}{ccc} \hat{x} & \hat{y} & \hat{z} \\ \frac{\partial}{\partial x} & \frac{\partial}{\partial y} & \frac{\partial}{\partial z} \\ 0 & H_y & 0 \end{array} \right| = - \dfrac{\partial H_y}{\partial z} \hat{x}
\label{Ampere's Law-Curl-Expansion}
\end{equation}
Rearranging terms,
\begin{equation}
\centering
\dfrac{\partial H_y}{\partial t} = - \dfrac{1}{\mu} \dfrac{\partial E_x}{\partial z}
\label{Faraday's Law-Curl-Expansion-1D}
\end{equation}
\begin{equation}
\centering
\dfrac{\partial E_x}{\partial t} = - \dfrac{1}{\epsilon} \dfrac{\partial H_y}{\partial z}
\label{Ampere's Law-Curl-Expansion-1D}
\end{equation}
These two scalar equations drive the FDTD algorithm. First, magnetic field is calculated using equation~\ref{Faraday's Law-Curl-Expansion-1D} and then electric field (equation~\ref{Ampere's Law-Curl-Expansion-1D}) is calculated from the magnetic field obtained in~\ref{Faraday's Law-Curl-Expansion-1D}. Electric and magnetic fields are staggered at half time steps. Similarly, magnetic and electric fields in space are also spaced at half spatial steps. The algorithm is normally referred to as a leap--frog scheme.

By examining equations~\ref{Faraday's Law-Curl-Expansion-1D} and \ref{Ampere's Law-Curl-Expansion-1D} it can be intuitively observed that a wave with $E_x$ and $H_y$ components will propagate in $\hat{z}$ direction and space derivatives with respect to $z$ indicate variation of electric and magnetic fields in space along $z$-axis.

These equations need to be discretized before they can be implemented as a computer simulation. Both the time and space derivatives are discretized to obtain difference equations. Following the treatment in chapter 3 of~\cite{JBSchneiderUFDTD}, electric and magnetic fields as functions of time and space can be written as,
\begin{equation}
\centering
H_y (z,t)=H_y (k\Delta _z,n\Delta _t)=H^n_y[k]
\label{Hy-of-z_and_t}
\end{equation}
\begin{equation}
\centering
E_x (z,t)=E_x (k\Delta _z,n\Delta _t)=E^n_x[k]
\label{Ex-of-z_and_t}
\end{equation}
The final equations for electric and magnetic fields driving the FDTD algorithm are,
\begin{equation}
\centering
H^{n+\frac{1}{2}}_y \left[k+\frac{1}{2}\right]=H^{n-\frac{1}{2}}_y \left[k+\frac{1}{2}\right] + \dfrac{\Delta t}{\mu \Delta x}
\label{Hy-1D-Simple-FDTD-Driver}
\end{equation}


\section{Stability and Causality}

The FDTD method is constrained by the stability relation known as the Courant stability criterion.

\section{Dispersive Models}
Debye, Drude and Lorentz dispersion models.

\section{Using the Drude Dispersive Model}
Using Drude model values of plasma frequencies for regions corresponding to given values of $\epsilon$ and $\mu$ are specified. The FDTD methods requires some modification~\cite{Radial-Zhao}.

\chapter{Problem Specification}

\begin{figure}[here]
\centering
%\includegraphics[width=\textwidth]{Algorithm.png}
\caption{The Yee algorithm. Figure taken from~\cite{JBSchneiderUFDTD}}
\label{Algorithm}
\end{figure}

\chapter{Implementation Details}


\chapter{Conclusion}

\nocite{*}
\bibliographystyle{ieeetr} %plain, ieeetr
\bibliography{FDTDref}

\end{document}




